\documentclass[12pt]{report}

\usepackage{geometry}
\usepackage[T1]{fontenc}
\usepackage[utf8]{inputenc}
\usepackage[francais]{babel}

\usepackage{graphicx}
\usepackage{lmodern}

\geometry{margin=2cm}


\begin{document}

\thispagestyle{empty}
\noindent\includegraphics[width=0.25\textwidth]{enseirb-matmeca}

\vspace{\stretch{1}}

\begin{center}
	\Huge{\textbf{Rapport de projet SGBD :}}

	\Huge{\textbf{Bandes dessinées}}
\end{center}

\vspace{\stretch{2}}

\begin{tabular}{r@{:~}l}
	\textbf{Auteurs} & \textit{David Bitonneau, Ludovic Hofer, Benoît Ruelle}\\
 \textbf{Encadrants} & \textit{Mme. Allyx Fontaine, M. Sylvain Lombardy,
M. Mohamed Mosbah}\\
\end{tabular}

\vspace{\stretch{1}}

\begin{center}Deuxième année, filière informatique

	Date : \today
\end{center}

\newpage

\section{Introduction}

\emph{re-decrire le sujet en detaillant}

L'objectif du projet est de mettre en œuvre, sur un cas pratique, les notions
et les méthodes vues dans le module de SGBD. L'application effectuée ici est
liée à la gestion de bandes dessinées.

Dans ce rapport, la base de données est décrite de sa conception à son
utilisation, en passant par son implémentation. Les choix effectués par le
groupe de projet sont détaillés et justifiés lorsque cela est nécessaire.

\section{Modélisation des données}

\emph{Justifier vos choix/hypotheses avec du texte}

\subsection{Description du contexte de l'application (entités, associations, règles de gestion)}

\paragraph{}
L'application doit permettre la gestion de bandes dessinées à partir des
informations suivantes :

\begin{itemize}
	\item Chaque volume de bande dessinée est soit un album, soit une revue.
	\item Tout volume a un éditeur, et une année d’édition.
	\item Un album peut éventuellement appartenir à une collection, et dans ce cas, il
		peut avoir un numéro dans cette collection. Deux albums de la même collection
		ont forcément le même éditeur.
	\item Un volume a un titre, qui est soit le titre de l'album, soit celui de la
		revue ; dans le cas de la revue, elle a aussi un numéro.
	\item Un volume peut contenir plusieurs histoires.
	\item Chaque histoire a un titre et une année de (première) parution ; elle a un
		ou plusieurs auteurs, chacun de ces auteurs s’occupant du dessin ou du
		scénario (ou des deux).
	\item Une même histoire peut apparaître dans différents volumes ; on veut pouvoir
		annoter la présence d’une histoire dans un volume ("Première publication",
		"Pages 15 à 22", "Version longue", etc.).
	\item Une histoire peut appartenir à une série et dans ce cas elle peut avoir un
		numéro de série.
\end{itemize}

\paragraph{}
La lecture de ces informations fait ressortir les entités suivantes :
\begin{itemize}
	\item volume ;
	\item album ;
	\item revue ;
	\item éditeur ;
	\item collection ;
	\item histoire ;
	\item série ;
	\item auteur.
\end{itemize}

\paragraph{Les hypothèses suivantes sont effectuées :}
\begin{itemize}
	\item On considère qu'une revue est un ensemble de numéros de la revue et que
		l'éditeur d'une revue peut varier dans le temps.
	\item Plusieurs collections différentes peuvent avoir le même nom.
\end{itemize}

\paragraph{}
Le schéma entité-association suivant montre les relations entre ces
différentes entités :

\noindent\includegraphics[width=\textwidth]{schema-entite-association}

\subsection{Opérations possibles sur la base}

Différentes opérations peuvent être effectuées sur la base de données.
Ces opérations peuvent être classées en 3 différentes catégories :
\begin{itemize}
	\item les opérations de consultation ;
	\item l'extraction de statistiques ;
	\item les mises à jour de la base.
\end{itemize}

\paragraph{Consultation}
Ces opérations permettent de fournir des informations basiques concernant
diverses entités de la base de données à partir des informations qui y sont
stockées.
Les informations suivantes sont notamment fournies :
\begin{itemize}
	\item la bibliographie d’un auteur, soit par ordre chronologique, soit classé par séries, en tant que
		scénariste, dessinateur ou auteur complet, en indiquant ses co-auteurs et leurs rôles.
	\item la liste des auteurs collaborant à une revue durant une certaine période, avec le nombre de
		numéros auxquels ils ont participé ;
	\item l'historique de la publication d’une histoire ;
	\item les histoires différentes ayant le même titre.
\end{itemize}

\paragraph{Statistiques}
Les statistiques suivantes peuvent être prélevées grâce aux requêtes
implémentée :
\begin{itemize}
	\item le nombre d’histoires auxquelles un auteur a participé ;
	\item la série ayant le plus grand nombre d’auteurs ;
	\item les histoires classées selon le nombre de fois où elles apparaissent en album ;
	\item le nombre moyen d’auteurs participant à une revue pendant une période donnée.
\end{itemize}
En plus de ces statistiques, il est possible de classer les auteurs selon la
décade durant laquelle ils ont créé le plus d’histoires (1910-1919, 1920-1929,
etc.).

\paragraph{Mise à jour}
Il est possible d'appliquer les opérations de modification de la tables
suivantes :
\begin{itemize}
	\item Ajout, suppression, modification d’une histoire, d’un volume, d’un auteur.
	\item Création d'un volume en spécifiant une histoire et son éditeur : ce
		volume contient alors l'histoire indiquée ; son titre et son année
		d’édition sont respectivement le titre et l'année de parution de
		l'histoire.
\end{itemize}

\paragraph{}
Les requêtes correspondant à ces opérations sont décrites plus bas.

\section{Schéma relationnel :}

\subsection{Passage au relationnel}

\subsection{Contraintes d'intégrité, dépendances fonctionnelles}

\subsection{Schéma relationnel en 3e forme normale}

\noindent\includegraphics[width=\textwidth]{schema-relation}

\section{Implémentation }
-> choix de MySQL

\subsection{Création de la base de données, en prenant en compte les contraintes d'intégrité (scripts de création, suppression, insertion)}

\subsection{Implémentation des commandes SQL réalisant les opérations retenues}

\emph{Lister les requetes importantes}

\section{Utilisation :}

\subsection{Description de l'environnement d'exécution}

\emph{Decrire l'environnement et l'installation}

\subsection{Notice d'utilisation}

La base est destinée à MySQL et peut être initialisée avec :
\begin{verbatim}
mysql> create database nom_de_la_bdd;
mysql> use nom_de_la_bdd;
mysql> source base.mysql;
mysql> source insertion.mysql;
\end{verbatim}

Il est alors nécessaire d'éditer le fichier include.php pour mettre le contenu
de la variable \verb!$bdd! avec \verb!nom_de_la_bdd! dans la fonction
\verb!connectdb()!. Si nécessaire, le contenu des variables \verb!$user! et
\verb!$passwd! utilisés pour la connexion à la base peut être changé au même
endroit.

\subsection{Description des interfaces éventuelles (Shell, JDBC, PHP, etc.)}

\end{document}
